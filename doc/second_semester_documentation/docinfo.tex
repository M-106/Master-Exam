% -------------------------------------------------------
% Daten für die Arbeit
% Wenn hier alles korrekt eingetragen wurde, wird das Titelblatt
% automatisch generiert. D.h. die Datei titelblatt.tex muss nicht mehr
% angepasst werden.

\newcommand{\hsmasprache}{en} % de oder en für Deutsch oder Englisch
% Für korrekt sortierte Literatureinträge, noch preambel.tex anpassen
% und zwar bei \usepackage[main=ngerman, english]{babel},
% \usepackage[pagebackref=false,german]{hyperref}
% und \usepackage[autostyle=true,german=quotes]{csquotes}

% Titel der Arbeit auf Deutsch
\newcommand{\hsmatitelde}{Können generative Modelle komplexe Zusammenhänge lernen?}

% Titel der Arbeit auf Englisch
\newcommand{\hsmatitelen}{Can generative Models learn Complex Relations?}

% Weitere Informationen zur Arbeit
\newcommand{\hsmaort}{Offenburg}    % Ort
\newcommand{\hsmaautorvname}{Tobia} % Vorname(n)
\newcommand{\hsmaautornname}{Ippolito} % Nachname(n)
\newcommand{\hsmadatum}{12.03.2027} % Datum der Abgabe
\newcommand{\hsmajahr}{2026} % Jahr der Abgabe
\newcommand{\hsmafirma}{IMLA, Herrenknecht Vertical GmbH and "KI-Bohrer" Projekt} % Firma bei der die Arbeit durchgeführt wurde
\newcommand{\hsmabetreuer}{Prof. Dr. rer. nat. Keuper Janis, Hochschule Offenburg} % Betreuer an der Hochschule
\newcommand{\hsmazweitkorrektor}{Martin Spitznagel, Hochschule Offenburg} % Betreuer im Unternehmen oder Zweitkorrektor
\newcommand{\hsmafakultaet}{EMI} % Fakultät
\newcommand{\hsmastudiengang}{MAR} % Studiengangsabkürzung. 
% Diese wird in titelblatt.tex definiert. Bisher AI, EI, MK und INFM. Bitte ergänzen.

% Zustimmung zur Veröffentlichung
\setboolean{hsmapublizieren}{true}   % Einer Veröffentlichung wird zugestimmt
\setboolean{hsmasperrvermerk}{false} % Die Arbeit hat keinen Sperrvermerk

% -------------------------------------------------------
% Abstract

% Kurze (maximal halbseitige) Beschreibung, worum es in der Arbeit geht auf Deutsch
\newcommand{\hsmaabstractde}{We introduce our Image-Physics-Simulation Framework, for 2D Image ray-tracing (classical and Image Source Method) to enrich the input of deep learning models with additional physical information to predict reflections evaluted on the PhysicsGen Benchmark for noisedata prediction in urban areas.\\
While classical ray-tracing seem to be not effective for the prediction of reflection, we see potential in ISM ray-traces based on a F1-score with the ground truth but without a model training yet.\\
Additional experiments show that pre and post masking does not help learning complex relationships on the PhysicsGen dataset.}

% Kurze (maximal halbseitige) Beschreibung, worum es in der Arbeit geht auf Englisch

\newcommand{\hsmaabstracten}{Englische Version von Lorem ipsum dolor sit amet, consetetur sadipscing elitr, sed diam nonumy eirmod tempor invidunt ut labore et dolore magna aliquyam erat, sed diam voluptua. At vero eos et accusam et justo duo dolores et ea rebum. Stet clita kasd gubergren, no sea takimata sanctus est Lorem ipsum dolor sit amet. Lorem ipsum dolor sit amet, consetetur sadipscing elitr, sed diam nonumy eirmod tempor invidunt ut labore et dolore magna aliquyam erat, sed diam voluptua. At vero eos et accusam et justo duo dolores et ea rebum. Stet clita kasd gubergren, no sea takimata sanctus est Lorem ipsum dolor sit amet.}
