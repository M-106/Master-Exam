\chapter{Introduction}
\label{cha:intro}

	Advances in deep neural networks have significantly improved representation learning for high-dimensional and structured data \cite{Goodfellow-et-al-2016, lecun_deep_2015}.\\
	Despite these successes, accurately predicting complex systems (those influenced by multiple interacting factors) remains an open challenge \cite{martin_spitznagel_physicsgen_2025}. This is particularly true for physical systems, where non-trivial interactions of multiple factors often require highly detailed and computationally expensive simulations due to the complexity of the underlying equations, such as partial differential equations (PDEs). However, future applications in robotics and manufacturing demand precise and fast predictions, which traditional simulations cannot provide.
	
	One important task in this context is the \textbf{paired Image-to-Image translation} \cite{isola_image--image_2016}, as it enables learning a direct mapping between corresponding simulated and target physical states under identical boundary conditions, represented in image form, a common abstraction in robotics and related applications. \\
	In contrast to the unpaired tasks, paired translation has access to a single ground-truth output for each input image. This paradigm is expected to grow in relevance as robotic systems become increasingly integrated into everyday environments and often work with camera-systems. A particularly interesting subarea is simulation distillation, where deep-learning models learn from simulated data to achieve the precision of those simulations and additionally have the speed required for real-time applications.
	
	In this work, we build upon our previous studies \cite{first_semester_report}, continuing our focus on predicting noise propagation generated by deep drilling operations in urban areas; an important topic for enabling climate-friendly geothermal energy and hot-water systems. Recent research \cite{martin_spitznagel_physicsgen_2025} has shown that the complex components of sound propagation remain difficult to predict; exhibiting, for example, over 2.5× higher L1 loss in non-line-of-sight (NLoS) regions than in line-of-sight (LoS) regions; highlighting the need for improved methods to learn physically complex features in paired image-to-image translation. Our work directly extends this line of research by using the established benchmark "PhysicsGen" \cite{martin_spitznagel_physicsgen_2025} to evaluate and compare different modelling approaches and methods.
	
	%(3.51-0.91 + 4.79-2.14)/2
	%2.625
	
	\clearpage
	
	% Our experiments differ from previous work due to the large number and diversity of approaches evaluated to answer our research questions.\\
	We revisit previous methods in practice and introduce a ray-tracing framework to investigate the effect of additional physical information. We further employ a swapping sample masking technique to study its influence on the complex image-generation task. 
	
	The following research questions guide this study:
	\begin{enumerate}[itemsep=0pt, topsep=0pt, parsep=0pt, partopsep=0pt]
		\item Having identified effective hyperparameters for reflection‑only prediction on a simpler dataset, I now aim to evaluate whether these findings generalize to a more complex dataset.
		\item What is the effect of pre- and post-masking on the prediction of physically complex regions?
		\item How does the integration of ray-tracing information influence model performance?
	\end{enumerate}
	These questions are designed to address known limitations of current learning-based approaches and to assess whether deep learning models can accurately capture complex physical interactions in an image-to-image setting.
	
	We hypothesize that incorporating additional physical information improves the prediction of physically complex regions in paired image-to-image translation. Specifically, we expect ray-tracing information to enhance model performance in regions governed by reflections, and masking strategies to focus learning on high-complexity areas. 
	
	%(In future add results here -> results are not there yet)\\
	%Our investigations reveal …
	
	
	
	
	% OLD: 
	
	% Your project is to find ways to improve the learning of complex relationships / sound-propagation.
	
	% add sub chapters?
	
	%Neural networks experienced a remarkable rise over the past decade, accompanied by rapid technological development. They have demonstrated impressive capabilities across a wide range of tasks and expanded into many new fields and applications.
	
	%Despite these advantages, predicting complex systems (systems influenced by multiple interacting factors) remains an open challenge. This is particularly true for physical systems, where non-trivial interactions of multiple factors often require highly detailed and computationally expensive simulations. However, future applications in robotics and manufacturing demand precise and fast predictions, which traditional simulations cannot provide.
	
%	One important task is the paired Image-to-Image translation. In contrast to the unpaired tasks, paired translation has access to a single ground-truth output for each input image. This paradigm is expected to grow in relevance as robotic systems become increasingly integrated into everyday environments and often work with camera-systems. A particularly interesting subarea is simulation distillation, where models learn from simulated data to achieve the precision and speed required for real-time applications.
	
	%In this work, we build upon our previous studies and focus on the prediction of noise generated by deep drilling operations in urban areas; an important topic for enabling climate-friendly geothermal energy and hot-water systems. Recent research has shown that the complex components of sound propagation remain difficult to predict, suggesting a need for improved methods for learning physically complex features within paired Image-to-Image translation. Our work directly extends this line of research by using the established benchmark (“PhysicsGen”) to evaluate and compare different modelling approaches and methods.
	
	%Our experiments differ from previous work due to the large number and diversity of approaches evaluated to answer our research questions.\\
	%We revisit previous methods in practice and introduce a ray-tracing framework to investigate the effect of additional physical information. We further employ a swapping sample masking technique to study its influence on the complex image-generation task. Additionally, we train a modern transformer architecture (“Cosmos Predict 2 WorldModel”, introduced by a researcher team at NVIDIA) to assess its performance and capabilities on the PhysicsGen benchmark.
	
	%\section{Research Questions}
	%	\label{sec:method-research-question}
	%	This study investigates the following:
	%	\begin{enumerate}[itemsep=0pt, topsep=0pt, parsep=0pt, partopsep=0pt]
	%		\item Performance of a fine-tuned Residual Design Pix2Pix Model.
	%		\item Influence of pre- and post-masking on predicting physically complex regions.
	%		\item Effect of integrating ray-tracing information as an additional model input.
	%		\item Performance of a large transformer-based world-model for 2D wave-propagation prediction.
	%	\end{enumerate} 
	%	These questions directly target known weaknesses of existing learning-based methods on the PhysicsGen benchmark and answer the global question of whether deep learning models can learn complex relationships in a image-to-image scenario.
	
	%\section{Topic \& Context}
	%\label{sec:intro-context}
	%	Introduction into the research field (“Paired Image-to-Image Translation”)
		
	%	Introduction into the research field (“Paired Image-to-Image Translation + learning complex relationships”) 
		
		
	%	Why relevant / challenging?
		
	
	%	Relation to real application / practical problems
		
	
	
	%\section{Problem \& Research-Gap}
	%\label{sec:intro-problem}
	%	Which challenges remain? (in the field)
		
		
	%	Knowledge we have from existing works
		
		
	%	What does we not know -> gap
		
		
	%	Why this gap is relevant?
		
		
		
	%\section{Goal}
	%\label{sec:intro-goal}
	%	Gaps handled by this work
		
		
	%	Scientific Question / Thesis of this work
		
		
	%	What should be answered/developed at the end?
		
		
		
	%\section{Contribution}
	%\label{sec:intro-contribution}
	%	Quickly/summary: Methods, Architectures/Models, Data Experiements are used?
		
		
	%	What is new about this approach?
		
		
	%	Results of the approach
		
		
		
	%\section{Structure of the work}
	%\label{sec:structure-of-work}
	%	...
		
		
		
		
		
		
		
		